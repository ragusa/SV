%% Based on a TeXnicCenter-Template, which was
%% created by Christoph B�rensen
%% and slightly modified by Tino Weinkauf.
%%%%%%%%%%%%%%%%%%%%%%%%%%%%%%%%%%%%%%%%%%%%%%%%%%%%%%%%%%%%%

\documentclass[12pt]{scrartcl} %This is a special class provided by the KOMA script, which does a lot of adjustments to adapt the standard LaTeX classes to european habits, change to [a4paper,12pt,twoside] for doublesided layout


%########################### Preferences #################################


% ******** vmargin settings *********
\usepackage{vmargin} %This give you full control over the used page area, it maybe not the idea od Latex to do so, but I wanted to reduce to amount of white space on the page
%\setpapersize{A4}
%\setmargins{3.5cm}%			%linker Rand, left edge
					 %{1.5cm}%     %oberer Rand, top edge
           %{14.7cm}%		%Textbreite, text width
           %{23.42cm}%   %Texthoehe, text hight
           %{14pt}%			%Kopfzeilenh�he, header hight
           %{1cm}%   	  %Kopfzeilenabstand, header distance
           %{0pt}%				%Fu�zeilenhoehe footer hight
           %{2cm}%    	  %Fusszeilenabstand, footer distance         

% ********* Font definiton ************
%\usepackage{t1enc} % as usual
%\usepackage[latin1]{inputenc} % as usual
%\usepackage{times}		
%\usepackage{mathptmx}  	%mathematical fonts for use with times, I encountered some problems using this package togather with pdftex, which I was not able to resolve

% ********* Graphics definition *******
%\usepackage[pdftex]{graphicx} % required to import graphic files
\usepackage{color} %allows to mark some entries in the tables with color
%\usepackage{eso-pic} % these two are required to add the little picture on top of every page
%\usepackage{everyshi} % these two are required to add the little picture on top of every page
%\renewcommand{\floatpagefraction}{0.7} %default:0.5 allows two big pictures on one page

\usepackage{amsthm}
\usepackage{amssymb}
\usepackage{amsmath}
\usepackage{amsfonts}
\usepackage{amstext}
\usepackage{amsbsy}
\usepackage{mathbbol} 
%\usepackage{pmatrix} 

%********** Enybeling Hyperlinks *******
%\usepackage[pdfborder=000,pdftex=true]{hyperref}% this enables jumping from a reference and table of content in the pdf file to its target

% ********* Table layout **************
\usepackage{booktabs}	  	%design of table, has an excellent documentation
%\usepackage{lscape}			%use this if you want to rotate the table together with the lines around the table

%  new definitions
\renewcommand{\div}{\bs{\nabla}\! \cdot \!}
\newcommand{\grad}{\bs{\nabla}}
% extra space
\newcommand{\qq}{\quad\quad}
% common reference commands
\newcommand{\eqt}[1]{Eq.~(\ref{#1})}                     % equation
\newcommand{\fig}[1]{Fig.~\ref{#1}}                      % figure
\newcommand{\tbl}[1]{Table~\ref{#1}}                     % table
\newcommand{\sct}[1]{Section~\ref{#1}}                   % section
\newcommand{\app}[1]{Appendix~\ref{#1}}                   % appendix

\newcommand{\bs}[1]{\mathbf{#1}}
\newcommand{\dd}{\mathrm{d}}

\newcommand{\be}{\begin{equation}}
\newcommand{\ee}{\end{equation}}
\newcommand{\vn}{\vec{n}}
\newcommand{\vel}{\vec{\mathrm{v}}}

% ********* Caption Layout ************
\usepackage{ccaption} % allows special formating of the captions
\captionnamefont{\bf\footnotesize\sffamily} % defines the font of the caption name (e.g. Figure: or Table:)
\captiontitlefont{\footnotesize\sffamily} % defines the font of the caption text (same as above, but not bold)
\setlength{\abovecaptionskip}{0mm} %lowers the distace of captions to the figure


% ********* Header and Footer **********
% This is something to play with forever. I use here the advanced settings of the KOMA script

\usepackage{scrpage2} %header and footer using the options for the KOMA script
\renewcommand{\headfont}{\footnotesize\sffamily} % font for the header
\renewcommand{\pnumfont}{\footnotesize\sffamily} % font for the pagenumbers

%the following lines define the pagestyle for the main document
\defpagestyle{cb}{%
(\textwidth,0pt)% sets the border line above the header
{\pagemark\hfill\headmark\hfill}% doublesided, left page
{\hfill\headmark\hfill\pagemark}% doublesided, right page
{\hfill\headmark\hfill\pagemark}%  onesided
(\textwidth,1pt)}% sets the border line below the header
%
{(\textwidth,1pt)% sets the border line above the footer
{{\it Jean Ragusa}\hfill TAMU}% doublesided, left page
{Jean Ragusa\hfill{\it TAMU}}% doublesided, right page
{Jean Ragusa\hfill{\it TAMU}} % one sided printing
(\textwidth,0pt)% sets the border line below the footer
}

%this defines the page style for the first pages: all empty
\renewpagestyle{plain}%
	{(\textwidth,0pt)%
		{\hfill}{\hfill}{\hfill}%
	(\textwidth,0pt)}%
	{(\textwidth,0pt)%	
		{\hfill}{\hfill}{\hfill}%
	(\textwidth,0pt)}

%********** Footnotes **********
\renewcommand{\footnoterule}{\rule{5cm}{0.2mm} \vspace{0.3cm}} %increases the distance of footnotes from the text
\deffootnote[1em]{1em}{1em}{\textsuperscript{\normalfont\thefootnotemark}} %some moe formattion on footnotes

\renewcommand{\labelitemii}{$\diamond$}

%%%%%%%%%%%%%%%%%%%%%%%%%%%%%%%% boxes
\usepackage{color}
\definecolor{myblue}{rgb}{.8, .8, 1}
\usepackage{empheq}

\newlength\mytemplen
\newsavebox\mytempbox

\makeatletter
\newcommand\mybluebox{%
    \@ifnextchar[%]
       {\@mybluebox}%
       {\@mybluebox[0pt]}}

\def\@mybluebox[#1]{%
    \@ifnextchar[%]
       {\@@mybluebox[#1]}%
       {\@@mybluebox[#1][0pt]}}

\def\@@mybluebox[#1][#2]#3{
    \sbox\mytempbox{#3}%
    \mytemplen\ht\mytempbox
    \advance\mytemplen #1\relax
    \ht\mytempbox\mytemplen
    \mytemplen\dp\mytempbox
    \advance\mytemplen #2\relax
    \dp\mytempbox\mytemplen
    \colorbox{myblue}{\hspace{1em}\usebox{\mytempbox}\hspace{1em}}}

\makeatother

%################ End Preferences, Begin Document #####################

\pagestyle{plain} % on headers or footers on the first page

%%%%%%%%%%%%%%%%%%%%%%%%%%%%%%%%%%%%%%%%%%%%%%%%%%%%%%%%%%%%%%%%%%%%%%%%%%%%%
%%%%%%%%%%%%%%%%%%%%%%%%%%%%%%%%%%%%%%%%%%%%%%%%%%%%%%%%%%%%%%%%%%%%%%%%%%%%%
\begin{document}
%%%%%%%%%%%%%%%%%%%%%%%%%%%%%%%%%%%%%%%%%%%%%%%%%%%%%%%%%%%%%%%%%%%%%%%%%%%%%
%%%%%%%%%%%%%%%%%%%%%%%%%%%%%%%%%%%%%%%%%%%%%%%%%%%%%%%%%%%%%%%%%%%%%%%%%%%%%

\begin{center}

\begin{figure}[th]
    \centering
		%\includegraphics[width=10cm]{logo.jpg}
	\label{fig:logo}
\end{figure}

\vspace{2cm}
\vspace{2cm}

% There might be better solutions for the title page, giving all distances and sizes manually was simply the easiest solution

{\Huge\bf\sf Entropy Method Notes}

\vspace{.5cm}

{\Huge\bf\sf TAMU}

\vspace{.5cm}

%{\Huge\bf\sf to use with TeXnicCenter}

\vspace{2cm}

{\Large\bf\sf Jean C. Ragusa}
\vspace{2cm}

{\Large\bf\sf \today} %adds the current date

\vspace{\fill}

\tt{jean.ragusa@tamu.edu}

\end{center}
\newpage

%%The following loads the picture on top of every page, the numbers in \put() define the position on the page:
%\AddToShipoutPicture{\setlength\unitlength{0.1mm}\put(604,2522){\includegraphics[width=1.5cm]{logo.jpg}}}

\pagestyle{cb} % now we want to have headers and footers

\tableofcontents

\newpage


%%%%%%%%%%%%%%%%%%%%%%%%%%%%%%%%%%%%%%%%%%%%%%%%%%%%%%%%%%%%%%%%%%%%%%%%%%%%%
%%%%%%%%%%%%%%%%%%%%%%%%%%%%%%%%%%%%%%%%%%%%%%%%%%%%%%%%%%%%%%%%%%%%%%%%%%%%%
\section{Shallow Water Equations}
%%%%%%%%%%%%%%%%%%%%%%%%%%%%%%%%%%%%%%%%%%%%%%%%%%%%%%%%%%%%%%%%%%%%%%%%%%%%%
%%%%%%%%%%%%%%%%%%%%%%%%%%%%%%%%%%%%%%%%%%%%%%%%%%%%%%%%%%%%%%%%%%%%%%%%%%%%%


%%%%%%%%%%%%%%%%%%%%%%%%%%%%%%%%%%%%%%%%%%%%%%%%%%%%%%%%%%%%%%%%%%%%%%%%%%%%%
\subsection{Governing equations}
%%%%%%%%%%%%%%%%%%%%%%%%%%%%%%%%%%%%%%%%%%%%%%%%%%%%%%%%%%%%%%%%%%%%%%%%%%%%%

The Shallow Water (SW) equations are given in terms of the water height $h$ and
water charge $\bs{q}=h\bs{u}$, where $\bs{u}$ is the velocity of the column of water. The SW
are basically height-integrated continuity and momentum equations:

\begin{subequations}
\label{eq:sw}
%
\begin{equation}
\partial_t h + \div \bs{q} = 0
\end{equation}
%
\begin{equation}
\partial_t \bs{q} + \div (\bs{u} \otimes \bs{q}) + \frac 1 2 \grad (gh^2) = -gh \grad B
\end{equation}
%
\end{subequations}
where $B(\bs{r})$ is the bottom relief. 

In 1-D, we can write them as 
\begin{subequations}
\label{eq:sw}
%
\begin{equation}
\partial_t h + \partial_x q = 0
\end{equation}
%
\begin{equation}
\partial_t q + \partial_x \Big( \frac{q^2}{h} + \frac{gh^2}{2} \Big)=  -gh \partial_x B
\end{equation}
%
\end{subequations}

%%%%%%%%%%%%%%%%%%%%%%%%%%%%%%%%%%%%%%%%%%%%%%%%%%%%%%%%%%%%%%%%%%%%%%%%%%%%%
\subsection{Properties of the inviscid SWE in 1-D}
%%%%%%%%%%%%%%%%%%%%%%%%%%%%%%%%%%%%%%%%%%%%%%%%%%%%%%%%%%%%%%%%%%%%%%%%%%%%%

In quasi-linear form, we have
\begin{equation}
\begin{pmatrix}
h\\
q
\end{pmatrix}_t 
+
M(h,q)\begin{pmatrix}
h\\
q
\end{pmatrix}_x 
=
-gh \frac{dB}{dx}
%
\end{equation}
($B$ is assumed independent of time!)
where $M(h,q)$ is given by
\begin{equation}
M(h,q)=
\begin{pmatrix}
0                     & 1\\
-\frac{q^2}{h^2} + gh & \frac{2q}{h} 
\end{pmatrix}
\end{equation}

We can find the eigenvalues of $M$ easily
\begin{equation}
\begin{vmatrix}
-\lambda              & 1\\
-\frac{q^2}{h^2} + gh & \frac{2q}{h}-\lambda 
\end{vmatrix}
=
\begin{vmatrix}
-\lambda   & 1\\
-u^2 + gh  & 2u-\lambda 
\end{vmatrix}
=0
\end{equation}
yielding
\begin{equation}
\lambda^\pm = u \pm \sqrt{gh}
\end{equation}
%
with associated eigenvectors 
%
\begin{equation}
r^\pm = 
\begin{pmatrix}
1 \\ u +\pm \sqrt{\frac{g}{h}}
\end{pmatrix}
\end{equation}
Note that when $h=0$, the system looses its strictly hyperbolic nature.

\bigskip
The mathematical entropy is defined as (sum of kinetic and potential energies)
\begin{equation}
	\eta(h,q) = \frac 1 2 hu^2 + \frac 1 2 gh^2 = \frac{q^2}{2h} + \frac{gh^2}{2}
\end{equation}
and the associated entropy flux is
\begin{equation}
  \mathcal{G}(h,q) = u \Big( \eta + \frac{gh^2}{2} \Big) = \frac{q^3}{2h^2} + ghq \,.
\end{equation}
\underline{Note:} A variant for the definition of $\eta$ adds the quantity $ghB$ to the definition. 
This seems more correct for the entropy minimum principle. Then the entropy flux $\mathcal{G}$ is augmented by
$ghBu=gqB$.

We have the entropy inequality
\begin{equation}
\partial_t \eta + \partial_x \mathcal{G} \le 0
\end{equation}
%--------------------------------------------
\begin{proof}[{\tt  Entropy minimum principle for inviscid SWE}]
Multiply the continuity equation by $\eta_h=\partial_h \eta$ and the momentum equation by 
$\eta_q=\partial_q \eta$. Add them together. The entropy equality is obtained (after some algebra).

\bigskip
\noindent
\textcolor{red}{Actually, let's try this differently:}\\
We assume that the entropy is a function of height and velocity
\begin{equation}
	\eta(h,u) = \frac 1 2 hu^2 + \frac 1 2 gh^2 
\end{equation}
and we have
\[
\partial_h \eta = \frac 1 2 u^2 + gh  \quad \text{and} \quad
\partial_u \eta = h u 
\]
%
\begin{enumerate}
%
\item Re-write the continuity equation with $h$ and $u$ only:
\[
\boxed{
\partial_t h + u\partial_x h + h\partial_x u =0
}
\]
%
\item Re-write the momentum equation with $h$ and $u$ only:
\[
\underline{ \underline{u\partial_t h}} +h \partial_t u + \underline{\underline{u^2 \partial_x h}} + 2 hu \partial_x u+  gh \partial_x (h +B) =0
\]
The underlined terms are equal to: $-hu \partial_x u$. Thus,
\[
\boxed{
\partial_t u + u \partial_x u+  g \partial_x (h +B) =0
}
\]
%
\item Multiply the continuity equation by $\eta_h=\partial_h \eta$ and the momentum equation by 
$\eta_u=\partial_u \eta$ and add them together:
\begin{multline}
\eta_h \left( \partial_t h + u\partial_x h + h\partial_x u \right)
+
\eta_u \left( \partial_t u + u \partial_x u+  g \partial_x (h +B) \right) =\\
\partial_t \eta + u\partial_x \eta + \eta_h h\partial_x u + \eta_u   g \partial_x (h +B) = 0
\end{multline}
%
\item Actually, we want
\[
\partial_t \eta + \partial_x \mathcal{F} = \partial_t \eta + \partial_h \mathcal{F} \partial_x h + \partial_u \mathcal{F} \partial_x u  
\]
so we group terms from step \#3 in $\partial_x h$ and $\partial_x u$:
\[
\begin{array}{l|c}
\partial_x h: & \eta_h u  + \eta_u g\\ \hline
\partial_x u: & \eta_h h  + \eta_u u
\end{array}
\]
Thus
\[
\partial_h \mathcal{F} =  \eta_h u  + \eta_u g = 2ghu + \frac{u^3}{2}
\]
and
\[
\partial_u \mathcal{F} =  \eta_h h  + \eta_u u = gh^2 + \frac{3}{2}h u^2
\]
Integrating $\partial_h \mathcal{F}$ wrt $h$ gives
\[
\mathcal{F}(h,u) = gh^2u + \frac{hu^3}{2} + \alpha(u)
\]
Taking the derivative wrt $u$ and plugging in $\partial_u \mathcal{F}$ gives $\alpha'(u)=0$, or $\alpha(u)=C$.
\end{enumerate}
%
Thus,
\begin{equation}
\mathcal{F}(h,u) = gh^2u + \frac{hu^3}{2}
\end{equation}


\bigskip
For non smooth solutions, we follow Leveque ... to be finished ...
\end{proof}
%--------------------------------------------

\bigskip

The system should admit smooth steady-state solutions:
\begin{equation}
q=hu=\text{constant} \quad \text{and} \quad 
\frac{u^2}{2} + g(h+B)=\text{constant}
\end{equation}
as well as non-smooth steady-state solutions. The ``lake at rest'' is an important steady-state solution
\begin{equation}
u=0 \quad \text{and} \quad  g(h+B)=\text{constant}
\end{equation}

%--------------------------------------------
\begin{proof}[{\tt SW steady-state solution}]
Obviously, from the continuity equation, we have at steady state $\partial_x q =0$, that is, $q=hu=$constant.

Reporting in the momentum equation at steady state, we have
\[ 
-gh \partial_x B 
=
\partial_x \Big( \frac{q^2}{h} + \frac{gh^2}{2} \Big)
=
\partial_x \Big( hu^2 + \frac{gh^2}{2} \Big)
=
hu \partial_x u +gh\partial_x h
\]
That is, 
\begin{multline}
gh \partial_x (h+B) + hu \partial_x u =0 
\Longleftrightarrow g \partial_x (h+B) + u \partial_x u =0 \\
\Longleftrightarrow  \partial_x \Big(g(h+B) +\frac{u^2}{2} \Big) =0
\Longleftrightarrow  g(h+B) +\frac{u^2}{2} = \text{constant}
\end{multline}
The lake at rest solution is also easily found by setting $u=0$ in the above.
\end{proof}
%--------------------------------------------

%%%%%%%%%%%%%%%%%%%%%%%%%%%%%%%%%%%%%%%%%%%%%%%%%%%%%%%%%%%%%%%%%%%%%%%%%%%%%
\subsection{Characteristics in 1-D}
%%%%%%%%%%%%%%%%%%%%%%%%%%%%%%%%%%%%%%%%%%%%%%%%%%%%%%%%%%%%%%%%%%%%%%%%%%%%%

We assume $B=0$ for demonstration. Starting from 
\[
\partial_t u + u \partial_x u+  g \partial_x h =0
\]
we add $\pm c\partial_x u \mp c \partial_x u=0$:
\[
\partial_t u + (u \pm c) \partial_x u \mp c \partial_x u +  g \partial_x h = 0
\]
\underline{Notation:}
\[
\left. \frac{Du}{Dt} \right|_{u \pm c} = \partial_t u + (u \pm c) \partial_x u
\]

From the continuity equation, we retrieve $\partial_x u$ as:
\[
- h\partial_x u = \partial_t h + u\partial_x h 
\]
and get:
\[
\left. \frac{Du}{Dt} \right|_{u \pm c}  \pm c \frac{\partial_t h + u\partial_x h }{h} +  g \partial_x h = 0
\]

Recalling that $c^2 = gh $, we have
\[
c \frac{\partial_r h}{h} = gc \frac{\partial_r h}{gh} = \frac{\partial_r gh}{c} = \frac{\partial_r c^2}{c} = 2 \partial_r c
\]
and therefore
\[
\pm c \frac{\partial_t h + u\partial_x h }{h} =  \pm 2 \left(\partial_t c + u \partial_x c \right)
\]
Now, we look at the last term:
\[
g \partial_x h =  \partial_x gh = \partial_x c^2 = 2c \partial_x c
\]
Thus, we have
\[
\left. \frac{Du}{Dt} \right|_{u \pm c}  \pm 2 \left(\partial_t c + (u \pm c) \partial_x c \right) = 0 
\]
or
\[
\left. \frac{Du}{Dt} \right|_{u \pm c}  \pm 2 \left. \frac{Dc}{Dt} \right|_{u \pm c}  = 0 
\]
and the characteristics are:
\begin{equation}
du + 2 dc =0 
\end{equation}
and
\begin{equation}
du - 2 dc =0 
\end{equation}


%%%%%%%%%%%%%%%%%%%%%%%%%%%%%%%%%%%%%%%%%%%%%%%%%%%%%%%%%%%%%%%%%%%%%%%%%%%%%
\subsection{Characteristics in 2-D}
%%%%%%%%%%%%%%%%%%%%%%%%%%%%%%%%%%%%%%%%%%%%%%%%%%%%%%%%%%%%%%%%%%%%%%%%%%%%%

In 2D, we have

\be
\partial_t U + \partial_x F(U) + \partial_y G(U) = S
\ee
%
where
%
\[
U =
\begin{bmatrix}
h \\ hu  \\  hv
\end{bmatrix}
=
\begin{bmatrix}
h \\ q_x  \\  q_y
\end{bmatrix}
\]
%
%
\[
F =
\begin{bmatrix}
hu \\ hu^2 + gh^2/2  \\  h u v
\end{bmatrix}
=
\begin{bmatrix}
q_x \\ q_x^2/h + gh^2/2  \\  q_x q_y / h
\end{bmatrix}
\]
%
\[
G =
\begin{bmatrix}
hv \\  h u v\\ hv^2 + gh^2/2  
\end{bmatrix}
=
\begin{bmatrix}
q_y \\  q_x q_y / h\\ q_y^2/h + gh^2/2  
\end{bmatrix}
\]

%%%%%%%%%%%%%%%%%%%%%%%%%
\subsubsection{Conservative variable form}
%%%%%%%%%%%%%%%%%%%%%%%%%


In linearized form, the system becomes
\be
\label{eq:SV_linearized_form}
\partial_t U + A(U) \partial_x U + B(U) \partial_y U = S
\ee
%
where 
%
\[
A = \frac{ \partial F } { \partial U} =
\begin{bmatrix}
0               &  1         &   0 \\
-q_x^2/h^2 + gh &  2 q_x/h   &   0 \\
-q_x q_y /h^2   &  q_y/h     &  q_x/h
\end{bmatrix}
=
\begin{bmatrix}
0         &  1     &   0 \\
-u^2 + c^2&  2 u   &   0 \\
-u v      &  v     &   u
\end{bmatrix}
\]
%
\[
B = \frac{ \partial G } { \partial U} =
\begin{bmatrix}
0               &  0       &   1 \\
-q_x q_y /h^2   &  q_y/h   &  q_x/h \\
-q_y^2/h^2 + gh &  0       &  2 q_y/h
\end{bmatrix}
=   
\begin{bmatrix}
0         &  0   &   1 \\
-u v      &  v   &   u \\
-v^2 + c^2&  0   &   2v
\end{bmatrix}
\]

We then form the following matrix ($A$ and $B$ are diagonalisable but do not have the same eigenvectors, so we diagonalize a linear combination of $A$ and $B$):
\be
K = A n_x + B n_y
\ee
with $\vn=[n_x, n_y]^T$. 


\bigskip
The eigenvalues of $K$ are
\begin{align}
\lambda_1 &= \vel \cdot \vn \\
\lambda_2 &= \vel \cdot \vn +c \\
\lambda_3 &= \vel \cdot \vn -c
\end{align}
with $\vel = [u,v]^T$.

The right eigenvectors are
%
\[
R =
\begin{bmatrix}
  0   &   1        &    1 \\
-n_y  &  u+c n_x   &  u-c n_x \\
 n_x  &  v+c n_y   &  v-c n_y
\end{bmatrix}
\]
%
such that
\be
K R = R \Lambda 
\ee
where
%
\[
\Lambda =
\begin{bmatrix}
\lambda_1 &    0      &    0     \\
    0     & \lambda_2 &    0     \\
    0     &    0      &  \lambda_3
\end{bmatrix}
\]
%
The left eigenvectors are
%
\[
L^T =
\begin{bmatrix}
  u n_y - v n_x   &   c - \vel \cdot \vn   &  -c - \vel \cdot \vn   \\
-n_y              &   n_x                  &   n_x                  \\
 n_x              &   n_y                  &   n_y                  \\
\end{bmatrix}
\]
%
such that
\be
\label{eq:left_eigenproblem}
L K = \Lambda  L \quad \text{or} \quad K^T L^T = L^T \Lambda 
\ee

\bigskip
If everything were right so far, I should be able to have
\be
L R = I
\ee
but I have
\[
L R =
\begin{bmatrix}
   1     &    0     &    0     \\
   0     &   2c     &    0     \\
   0     &    0     &   2c
\end{bmatrix}
\]

However, since $L$ and $R$ are made of eignevectors, we can always multiply their columns by a constant. If I choose
\[
R       \longleftarrow R / 2c            \quad \text{and} \quad
L_{1,:} \longleftarrow L_{1,:} \times 2c \quad \text{and} \quad
L_{3,:} \longleftarrow L_{3,:} \times (-1)
\]
then I have $LR=I$. Note that the left eigenvectors (see \eqt{eq:left_eigenproblem}) are columns of $L^T$ this is why the rows of $L$ are multiplied by a constant. 

 
%%%%%%%%%%%%%%%%%%%%%%%%%
\subsubsection{Primitive variable form}
%%%%%%%%%%%%%%%%%%%%%%%%%

We start again from the linearized form, \eqt{eq:SV_linearized_form} but introduce the primitive variables
\[
V =
\begin{bmatrix}
h \\ u \\ v
\end{bmatrix}
\]
The idea is to use chain rule: $\frac{\partial U}{\partial x} = \frac{ \partial U}{\partial V} \frac{ \partial V}{\partial x}$

The Jacobian of this transformation is
\[
M =  \frac{ \partial U}{\partial V}  =
\begin{bmatrix}
   1      &    0     &    0     \\
    u     &    h     &    0     \\
    v     &    0     &    h
\end{bmatrix}
\]

Then, the linearized form becomes:
\be
M\partial_t V +  A M \partial_x V +  B M \partial_y V = S
\ee
Multiply throughout by $M^{-1}$, we get
\be
\label{eq:SV_linearized_form2}
\partial_t V + \tilde A  \partial_x V + \tilde B  \partial_y V = S
\ee
with
\[
\tilde A = M^{-1} A M 
=
\begin{bmatrix}
  u  &   h   &  0 \\
  g  &   u   &  0 \\
  0  &   0   &  u
\end{bmatrix}
%
\quad \text{and} \quad 
\tilde B = M^{-1} B M
=
\begin{bmatrix}
  v  &   0   &  h \\
  0  &   v   &  0 \\
  g  &   0   &  v
\end{bmatrix}
\]

As before, we introduce a linear combination of the matrices
\[
\tilde K = \tilde A n_x + \tilde B n_y
=
\begin{bmatrix}
\vel \cdot \vn &   h n_x            &  h n_y           \\
  g n_x        &   \vel \cdot \vn   &  0               \\
  g n_y        &   0                 & \vel \cdot \vn
\end{bmatrix}
\]
One can easily verify that $K$ and $\tilde K$ have the same eigenvalues. 

The right eigenvectors of $\tilde K$ are
\[
\tilde R =
\begin{bmatrix}
  0   &   h          &    h \\
-n_y  &  c n_x   &  -c n_x \\
 n_x  &  c n_y   &  -c n_y
\end{bmatrix}
\text{ or, equivalently, }
\begin{bmatrix}
  0   &   c          &    c \\
-n_y  &  g n_x   &  -g n_x \\
 n_x  &  g n_y   &  -g n_y
\end{bmatrix}
\]

The left eigenvectors $\tilde K$ are
%
\[
\tilde L^T =
\begin{bmatrix}
   0         &   c             &  -c  \\
-n_y       &  h n_x       &   h n_x                  \\
 n_x       &  h n_y       &   h n_y                  \\
\end{bmatrix}
\]

If everything were right so far, I should be able to have
\be
\tilde L \tilde R = I
\ee
but I have
\[
\tilde L \tilde R =
\begin{bmatrix}
    1     &    0      &    0     \\
    0     &   2ch     &    0     \\
    0     &    0      &   2c^2
\end{bmatrix}
\]

However, since $L$ and $R$ are made of eigenvectors, we can always multiply their columns by a constant. If I choose
\[
R_{:,2} \longleftarrow R_{:,2} / (2ch) \quad \text{and} \quad
R_{:,3} \longleftarrow R_{:,3} / (2c^2) 
\]
that is,
\[
\tilde R =
\begin{bmatrix}
  0   & 1/(2c)    &    1/(2g) \\
-n_y  & n_x/(2h)  &  n_x/(2c) \\
 n_x  & n_y/(2h)  &  n_y/(2c)
\end{bmatrix}
\]
then I have $\tilde L \tilde R=I$. 

\bigskip
Actually, there is another option that makes $\tilde R$ and $\tilde L$ look quite simple. Their expressions are:
\[
\tilde R =
\begin{bmatrix}
  0   & c/2     &   -c/2 \\
 n_y  & gn_x/2  &  gn_x/2 \\
-n_x  & gn_y/2  &  gn_y/2
\end{bmatrix}
\quad \text{and} \quad
\tilde L =
\begin{bmatrix}
  0   & n_y    &   -n_x \\
 1/c  & n_x/g  &  n_y/g \\
-1/c  & n_x/g  &  n_y/g
\end{bmatrix}
\]
We have $\tilde L \tilde R=I$.

\bigskip

The characteristic variable are introduced as follows
\[
 W = \tilde L V = \tilde R^{-1} V 
= 
\begin{bmatrix}
  0   & n_y    &   -n_x \\
 1/c  & n_x/g  &  n_y/g \\
-1/c  & n_x/g  &  n_y/g
\end{bmatrix}
\begin{bmatrix}
 h \\  u  \\  v
\end{bmatrix}
=
\begin{bmatrix}
  n_y  u   -n_x  v\\
 \frac{h}{c^2} \left(\vel \cdot \vn + c\right) \\
 \frac{h}{c^2} \left(\vel \cdot \vn - c\right) 
\end{bmatrix}
\]

The characteristic equation is
\be
\label{eq:SV_char1}
\partial_t W + \tilde L \tilde A \tilde R  \partial_x W + \tilde L \tilde B \tilde R \partial_y W = \tilde L S
\ee
or
\be
\partial_t W + \tilde R^{-1} \tilde A \tilde R  \partial_x W + \tilde R^{-1} \tilde B \tilde R \partial_y W = \tilde R^{-1} S
\ee

Next, we compute $\tilde L \tilde A \tilde R$ and decompose it in a diagonal part and a non-diagonal part:
\begin{multline}
\tilde L \tilde A \tilde R 
= 
\begin{bmatrix}
   u      & gn_y/2    &   -gn_y/2 \\
 h n_y/c  & u+c n_x   &   0 \\
-h n_y/c  & 0         &  u - c n_x
\end{bmatrix}
\\
=
\begin{bmatrix}
   u      & 0         &   0 \\
   0      & u+c n_x   &   0 \\
   0      & 0         &  u - c n_x
\end{bmatrix}
+
\begin{bmatrix}
   0      & gn_y/2    &   -gn_y/2 \\
 h n_y/c  &   0       &    0 \\
-h n_y/c  & 0         &    0
\end{bmatrix}
= D^x + C^x
\end{multline}

Similarly for $\tilde L \tilde B \tilde R$:
\begin{multline}
\tilde L \tilde B \tilde R 
= 
\begin{bmatrix}
   v      & -gn_x/2   &   gn_x/2 \\
-h n_x/c  & v+c n_y   &   0 \\
 h n_x/c  & 0         &  v - c n_y
\end{bmatrix}
\\
=
\begin{bmatrix}
   v      & 0         &   0 \\
   0      & v+c n_y   &   0 \\
   0      & 0         &  v - c n_y
\end{bmatrix}
+
\begin{bmatrix}
   0      & -gn_x/2   &   gn_x/2 \\
-h n_x/c  &   0       &    0 \\
 h n_x/c  & 0         &    0
\end{bmatrix}
= D^y + C^y
\end{multline}


We define $Q$
\[
Q = C^x \partial_x W + C^y \partial_x W
=
\begin{bmatrix}
g \left( n_y \partial_x h -n_x \partial_y h \right) \\
\frac{c}{g} \left( n_y^2 \partial_x u -n)x n_y ( \partial_y u\partial_x v) +n_x^2 \partial_y v \right)  \\
\frac{c}{g} \left( n_y^2 \partial_x u -n)x n_y ( \partial_y u\partial_x v) +n_x^2 \partial_y v \right)  
\end{bmatrix}
\]
The last 2 rows of $Q$ are identical.

If we introduce $\vec{N} = [n_y, -n_x]^T$, $Q=0$ (compatibility condition) is equivalent to:
\[
Q=0 \quad \Longleftrightarrow \quad
\begin{cases}
\vec{N} \cdot \grad h = 0 \\
\vec{N} \cdot (\vec{N} \cdot \div)\vel = 0
\end{cases}
\]

Finally, we have
\be
\label{eq:SV_char2}
\partial_t W + D^x  \partial_x W + D^y \partial_y W = \tilde L S
\ee
or
\begin{subequations}
\be
n_y \frac{D_1 u}{Dt} - n_x \frac{D_1 v}{Dt} 
\ee
\be
\frac{n_x}{g} \frac{D_2 u}{Dt} + \frac{n_y}{g} \frac{D_2 v}{Dt} + \frac{1}{c} \frac{D_2 h}{Dt} = ...
\ee
\be
\frac{n_x}{g} \frac{D_3 u}{Dt} + \frac{n_y}{g} \frac{D_3 v}{Dt} - \frac{1}{c} \frac{D_3 h}{Dt} = ...
\ee
\end{subequations}
where the material derivatives are as follows
\[
\frac{D_2 }{Dt} = \frac{\partial }{\partial t} + u \frac{\partial }{\partial x} + v \frac{\partial }{\partial y}
\]
%
\[
\frac{D_2 }{Dt} = \frac{\partial }{\partial t} + (u+n_xc) \frac{\partial }{\partial x} + (v+n_yc) \frac{\partial }{\partial y}
\]
and
\[
\frac{D_3 }{Dt} = \frac{\partial }{\partial t} + (u-n_xc) \frac{\partial }{\partial x} + (v-n_yc) \frac{\partial }{\partial y}
\]


%%%%%%%%%%%%%%%%%%%%%%%%%
% \subsubsection{Comparison of the two approaches}
%%%%%%%%%%%%%%%%%%%%%%%%%

%
%%%%%%%%%%%%%%%%%%%%%%%%%%%%%%%%%%%%%%%%%%%%%%%%%%%%%%%%%%%%%%%%%%%%%%%%%%%%%
\subsection{Viscous regularization of the SWE}
%%%%%%%%%%%%%%%%%%%%%%%%%%%%%%%%%%%%%%%%%%%%%%%%%%%%%%%%%%%%%%%%%%%%%%%%%%%%%

\begin{subequations}
\label{eq:sw}
%
\begin{equation}
\partial_t h + \partial_x q = \partial_x f
\end{equation}
%
\begin{equation}
\partial_t q + \partial_x \Big( \frac{q^2}{h} + \frac{gh^2}{2} \Big)=  -gh \partial B + \partial_x k
\end{equation}
%
\end{subequations}
where $f$ and $k$ are viscous fluxes. To obtain the entropy residual, recall that the governing 
equations were dotted by $[\eta_h\,,\eta_q]^T$.
This means that the extra term on the rhs of the entropy inequality is of the form
\[
rhs =\eta_h \partial_x f + \eta_q \partial_x k
=\Big( -\frac{q^2}{2h^2}+gh \Big) \partial_x f + \frac{q}{h}\partial_x k
=\big( -u^2+gh \big) \partial_x f + u\partial_x k
\]
Assuming 
\[ 
f = -\mu \partial_x h \quad \text{and} \quad k = -\mu \partial_x(hu)
\]
Then (omitting $\mu$ for brevity)
\begin{multline}
-rhs = \big( -u^2/2+gh \big) \partial_{xx} h + u\big( u\partial_{xx}h + 2 \partial_x h \partial_x u + h \partial_{xx}u  \big) \\
=\big( u^2/2+gh \big) \partial_{xx} h + \partial_x h \partial_x u^2 + hu \partial_{xx}u  
\end{multline}

%%%%%%%%%%%%%%%%%%%%%%%%%%%%%%%%%%%%%%%%%%%%%%%%%%%%%%%%%%%%%%%%%%%%%%%%%%%%%
%%%%%%%%%%%%%%%%%%%%%%%%%%%%%%%%%%%%%%%%%%%%%%%%%%%%%%%%%%%%%%%%%%%%%%%%%%%%%
\subsection{Implementation}
%%%%%%%%%%%%%%%%%%%%%%%%%%%%%%%%%%%%%%%%%%%%%%%%%%%%%%%%%%%%%%%%%%%%%%%%%%%%%

First-order viscosity
\[
\mu_{\max{}} = C_{\max{}} h \big(|u|+\sqrt{gh}\big)
\]
Entropy residual
\[
R= \partial_t \eta + \partial_x \mathcal{G}
\]
Entropy viscosity
\[
\mu_\text{ent}  = h^2 \frac{|R|}{\|\eta -\bar \eta \|_\infty}
\]

Viscosity
\[ \mu =\min \Big( \mu_{\max{}} , \mu_\text{ent}  \Big)
\]

Dissipative terms
\[ 
f = -\mu \partial_x h \quad \text{and} \quad k = -\mu \partial_x(hu)
\]

%%%%%%%%%%%%%%%%%%%%%%%%%%%%%%%%%%%%%%%%%%%%%%%%%%%%%%%%%%%%%%%%%%%%%%%%%%%%%
\subsection{Residual, jacobian}
%%%%%%%%%%%%%%%%%%%%%%%%%%%%%%%%%%%%%%%%%%%%%%%%%%%%%%%%%%%%%%%%%%%%%%%%%%%%%

Everything to the left-hand side:
\begin{subequations}
\label{eq:sw_weak_form}
%
\begin{equation}
\frac{d}{dt} \int \varphi_i  h - \int  \bs{q} \cdot \grad  \varphi_i = 0
\end{equation}
%
\begin{equation}
\frac{d}{dt} \int \varphi_i  \bs{q} - \int  (\frac{\bs{q}}{h} \otimes \bs{q} + \frac{gh^2}{2} )  \grad \varphi_i  = -  \int \varphi_i gh \grad B
\end{equation}
%
\end{subequations}

Steady state residual:

%%%%%%%%%%%%%%%%%%%%%%%%%%%%%%%%%%%%%%%%%%%%%%%%%%%%%%%%%%%%%%%%%%%%%%%%%%%%%


%%%%%%%%%%%%%%%%%%%%%%%%%%%%%%%%%%%%%%%%%%%%%%%%%%%%%%%%%%%%%%%%%%%%%%%%%%%%%
%%%%%%%%%%%%%%%%%%%%%%%%%%%%%%%%%%%%%%%%%%%%%%%%%%%%%%%%%%%%%%%%%%%%%%%%%%%%%
\newpage

\bibliographystyle{unsrt}   % this means that the order of references
			    % is determined by the order in which the
			    % \cite and \nocite commands appear
\bibliography{entropy-notes}  % list here all the bibliographies that
			     % you need. 

%%%%%%%%%%%%%%%%%%%%%%%%%%%%%%%%%%%%%%%%%%%%%%%%%%%%%%%%%%%%%%%%%%%%%%%%%%%%%
%%%%%%%%%%%%%%%%%%%%%%%%%%%%%%%%%%%%%%%%%%%%%%%%%%%%%%%%%%%%%%%%%%%%%%%%%%%%%
\end{document}
%%%%%%%%%%%%%%%%%%%%%%%%%%%%%%%%%%%%%%%%%%%%%%%%%%%%%%%%%%%%%%%%%%%%%%%%%%%%%
%%%%%%%%%%%%%%%%%%%%%%%%%%%%%%%%%%%%%%%%%%%%%%%%%%%%%%%%%%%%%%%%%%%%%%%%%%%%%



